\section*{Título Principal}

Bem-vindo ao documento de teste Markdown. Este documento inclui diversos elementos comuns do Markdown para ajudar na conversão para LaTeX.

\subsection*{Seção 1: Títulos}

\subsubsection*{Subtítulo 1.1}

\paragraph*{Subtítulo 1.1.1}

\subsection*{Seção 2: Ênfase}

Texto em \textit{itálico} e \textbf{negrito}.

Texto com ambos: \textbf{\textit{itálico e negrito}}.

\subsection*{Seção 3: Listas}

\subsubsection*{Lista não ordenada}

\begin{itemize}
	\item Item 1
	\begin{itemize}
		\item Subitem 1.1
		\item Subitem 1.2
	\end{itemize}
	\item Item 2
	\item Item 3
\end{itemize}

\subsubsection*{Lista ordenada}

\begin{enumerate}
	\item Primeiro item
	\item Segundo item
	\item Terceiro item
	\begin{enumerate}
		\item Subitem 3.1
		\item Subitem 3.2
	\end{enumerate}

\subsection*{Seção 4: Links e Imagens}
\end{enumerate}

[Link para o Google](https://www.google.com)

![Imagem de exemplo](https://via.placeholder.com/150)

\subsection*{Seção 5: Bloco de Código}

```python
def hello_world():
    print(``Hello, World!'')

hello_world()
```

\subsection*{Seção 6: Tabelas}

| Nome       | Idade | Cidade       |
|------------|-------|--------------|
| Alice      | 30    | São Paulo    |
| Bob        | 25    | Rio de Janeiro|
| Charlie    | 35    | Belo Horizonte|

\subsection*{Seção 7: Citações}

> ``A imaginação é mais importante que o conhecimento.'' - Albert Einstein

\subsection*{Seção 8: Fórmulas Matemáticas}

Aqui está uma fórmula matemática em linha: \( E = mc^2 \).

E uma fórmula em um bloco separado:

\[ 
a^2 + b^2 = c^2 
\]

\subsection*{Seção 9: Blocos de Código com Sintaxe Diferente}

```json
{
  ``nome'': ``John'',
  ``idade'': 30,
  ``cidade'': ``Nova York''
}
```

\subsection*{Seção 10: Texto Normal}

Este é um parágrafo comum. Ele pode conter \textbf{negrito}, \textit{itálico}, e até [links](https://www.example.com). Markdown é muito útil para escrever textos formatados de forma simples e rápida.

Fim do documento de teste.
